\section{Zusammenfassung}
Die erreichte Genauigkeit von $48.18 \, \%$ des CNN ist eine wesentlich höhere Genauigkeit als wir bei 12 Genre erwartet hätten.
Im Vergleich zu der Genauigkeit des KNN von nur $15 \, \%$ sagte das CNN wesentlich zuverlässiger das Genre der Bilder vorraus als der KNN.
Durch bessere Bildqualität und einen größeren Datensatz könnte wahrscheinlich diese Genauigkeit noch weiter gesteigert werden.
Da aber eine höhere Bildqualität eine höhere Pixelanzahl erfordert wäre der Datensatz schnell zu groß geworden um mit unseren Methoden verarbeitet werden zu können, da uns die Rechenleistung gefehlt hat. 
Zudem war es nicht optimal, das einige der Künstler*innen in mehreren Genre tätig waren und wir deswegen einigen Bilder des Datensatzes ein falschen Genre gegeben haben.
Denn uns hat die Zeit gefehlt um bei den besagten Künstlern*innen alle Bilder durch zu gehen und einzelnt zu entscheiden zu welchem Genre das Bild gehört.
Durch die falsche Etikettierung einiger Bilder ist die Genauigkeit des Models mit Sicherheit geringer geworden.
\\\\
Des weiteren ist der genutzte Datensatz für einige Genre nicht wirklich repräsentativ, da sich einige Genre zum größten Teil aus Bildern von nur einem Künstler*in zusammen setzten.
So besteht das Genre Post-Impressionismus zu über $71\,\%$ aus Bildern des Künstlers Vincent van Gogh.
Dies legt die Vermutung nahe, dass das Netzwerk bei einigen Genre eher den Stil eines bestimmten Künstlers*in erkennt, anstatt das Genre des Bildes.
Zu Umgehen wäre dieses Problem durch einen größeren Datensatz mit mehr Künstlern*innen pro Genre, die jeweils ähnlich viele Bilder geschaffen haben.
\\\\
Abschließend kann gesagt werden, dass das CNN zwar gute Ergebnisse liefert aber definitiv keinen Kunstkritiker*in ersetzt.
Dafür ist die Genauigkeit des Netzwerkes zu gering und die Kunst zu komplex.
Wobei dazu zu sagen ist, dass in den meisten Fällen Kunstkritikern*innen auch weitere Daten wie Erscheinungsdatum des Bildes und Nationalität des Künstlers*innen zu Verfügung stehen.
Durch diese könnte die Leistung des Netzwerkes ebenfalls verbessert werden, dies war allerdings nicht Teil unserer Problemstellung da wir allein aus den Bildern das Genre bestimmen wollten.
