\section{Motivation}
Kunst war schon immer ein großer Teil der Menschheitsgeschichte.
Durch Gemälde öffnet sich für den Betrachter*in eine Tür in die Gedanken des Künstlers*in.
Diese spiegeln sich besonders durch die verschiedenen Motive, die Künstler*in auf ihren Bildern verewigt haben wieder.
Aber auch die Darstellungsweise gibt einen Eindruck von der Zeit in der das Bild entstanden ist und aus welchem Umfeld der Künstler*in stammt.
So liegt es nur Nahe Bilder entsprechend ihre Motive, Darstellungsweise und nach ihrem Schaffungszeitpunkt zu kategorisieren.
In der Kunst werden diese Kategorien Genre oder Gattung genannt.
\\\\
Oft ist es allerdings schwer für den Laien*in die Einordnung eines Kunstwerkes in sein Genre selbstständig zu machen.
Denn einige Genre werden nur durch wenige Details von einem anderen Genre getrennt.
Teilweise verschwimmen diese Grenzen sogar komplett, wodurch dem Gemälde mehrere Genres zugesprochen werden können.
Es ist aber für das Verständnis und die Interpretation des Gemäldes essenziell das Genre des Bildes zu kennen.
Um die Kunst für Laien*in zugänglicher zu machen wird deshalb folgende Frage gestellt:
\\
\textbf{Ist es möglich das Genre eines Gemäldes durch das Einlesen des Bildes in ein Convolutional Neural Network (CNN) zu bestimmen?}
\\
Dies würde es dem Laien*in ermöglichen einfach das Genre eines Kunstwerkes zu erfahren und ihm so ein besseres Erlebnis beim Genuss der Kunst bieten.
Dabei haben wir uns zunächst auf einen Datensatz beschränkt, auf diesen wird im folgenden weiter eingegangen.
